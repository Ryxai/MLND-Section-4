\documentclass{article}
\usepackage{bibentry}
\usepackage{etoolbox}
\usepackage[]{geometry}
\usepackage[english]{babel}
\usepackage{mathpazo}
\usepackage{listings}
\usepackage{amsmath}
\usepackage{mathtools}
\usepackage{hyperref}
\makeatletter
\makeatother
\title{Reinforcement Learning}
\bibliographystyle{plain}
\author{}
\date{}
\pagestyle{empty}
\thispagestyle{empty}
\pagenumbering{gobble}
\setlength{\parindent}{0em}
\setlength{\parskip}{1em}
\begin{document}
\nocite{*}
\maketitle
\section{Markov Decision Processes}
Reinforcement learning is similar to supervised learning however the generated
outputs and their correctness are not the direct feedback given to the algorithm
so it can improve.

\subsection{Markov Decision Processes}

We begin with our states $S$ which represent the possible configurations of the
agent and the world.

Actions are things that can be done at a particular state (edges connecting
states). Given as $A$ or as $A(s)$ where $A(s)$ represents the actions of the
particular state $s$.

The model also known as the transition function or transition model, can be
considered the physics of the world. $T(s,a,s')$ gives us the probability
$\Pr(s'\mid s,a)$. That is the probability of transitioning to state $s'$ given
we began at state $s$ and took action $a$.

Markovian property is that only the present matters, we only have to condition
on the current state. We can turn any process into a Markovian process by
forcing each state to remember the required context. This is similar to the
method of turning a FSM into a DFSM.

The transition model is stationary and does not change.

Reward encompasses the domain knowledge and there are three main definitions
$R(s)$, $R(s,a)$ and $R(s,a,s')$. All of the variations are mathematically
equivalent, just indicate the specificity and granularity.

All of these factors together create a Markov Decision Problem.

A policy is the solution, that is given a state $s$ it generates the action to
take $a$. The policy is written $\pi(s) \rightarrow a$.

The policy that maximizes the reward function is written $\pi*$ and optimizes
the reward.

MDP does not require an explicit termination point.

MDP is plan-blind and only works from the current state of the world. It uses
policies which can be used to infer plans but reflects all possible states in
the world. Furthermore, it is resilient to stochasticity.

Delayed rewards are based on the idea that taking an immediate action does not
result in receiving a distinction whether that action was correct or not
immediately. That is there is a sequence of steps to be taken before a reward is
offered or denied.

Choosing which path in an MDP is the best is known as the temporal credit
assignment problem.

Setting the reward to a negative value except for the terminal states, results
in an optimization for the shortest path.

Minor changes to the reward function are important due to the multiplicative
impact of stochasticity.

Rewards can be considered the teaching signal and act as domain knowledge.

Infinite horizons are represented by MDP that allow for infinite sequences of
moves with no direct termination as a result of the number of moves taken.

With limited scope it is possible that risk/reward analysis results in different
policies for the same given state. This idea can be represented by $\pi(s,t)
\rightarrow a$. However this information can be encoded into the world directly
by mapping it into the physics. That is, the 'separate' aspect method does not
offer is not capable of a wider degree of representations, but that it may allow
for those representations to be encoded with greater accessibility and more
minimal scope/computation. (Consider FSM and DFSM).

The differential of the utility of sequences is independent of their common
prefix states. That is given two sequences $A, B$ where $A = (s_0, s_1, s_2,
\ldots)$ and $B = (s_0, s_1', s_2', \ldots)$ if $U(A) > U(B)$ then $U(A/s_0) >
U(B/s_0)$. This is referred to the stationary preferences. This follows from
adding rewards.

We can define the utility of a sequence of states is defined $U(s_0, s_1, s_2,
\ldots) = \sum_{t = 0}^\infty R(s_t)$. Infinite sequences however, degrade the
quality of this definition by making no difference between two functions that
both asymptotically trend towards infinity. We can change this definition to
reflect those circumstances, that is we can define $U(s_0, s_1, s_2) =
\sum_{t = 0}^\infty \gamma^tR(s_t)$ where $0 \leq \gamma < 1$. This forces the
reward function to converge over infinite steps. This is called discounted sums.

It is a geometric series that is bounded $U(S) \leq \sum_{t = 0}^\infty \gamma^t
R_{\text{max}} = \frac{R_{\text{max}}}{1 - \gamma}$. This is alternatively known
as a discounted series that is infinite in length but returns a finite result.

The optimal policy is defined $\Pi^* = \text{argmax}_\pi E[\sum_t \gamma^t
R(s_t) \mid \Pi ]$. The utility of a particular state, based on the optimal
policy is designated $U^*$ and the utility of a particular state is defined as
$U^*(s) = E[\sum_{t = 0}^{\inf} \gamma^t R(s_t) \mid \Pi, s_0 = s]$ which
says that the expected state of states that will be seen given that policy and
state.

The reward for entering a state $s_i$ defined $R(s_i) \neq U^*(S_i)$. Utility is
defined as long term feedback, whereas rewards are the short-term version. The
utility measures the future reward to be obtained given a specific policy, its a
measure of delayed reward.

The optimal policy at a given state is defined as $\Pi^*(s) = \text{argmax}_a
\sum_{s'}T(s,a,s)U(s')$ where $U(s) = U^*(s)$. Then $U(s) = R(s) + \gamma
\text{max}_a \sum_{s'} T(s,a,s') U(s')$, also known as the Bellman equation.

Solving for the Bellman equation is difficult, due to non-linearity as a result
solving based on the nearby utilities progressively is the solution. This update
equation is defined $\hat{U}_{t + 1}(s) = R(S) + \gamma \text{max}_a \sum_{s'}
T(s,as') \hat{U}_t(s')$. Using the Bellman update equation spreads out the
reward $R(s)$ via a \href
{https://stats.stackexchange.com/questions/243384/deriving-bellmans-equation-in-reinforcement-learning}
{contraction proof via value iteration}.

When considering the correct policy it is important to remember that the function
of $\Pi$ or the optimal policy is not a function of the utility $U$ but of the
rewards and state transitions. As a result, a policy may not get the utility
information of each state correct, but so long as the orer is correct (in
terms of states and transitions then this is sufficient).

We can then find the best policy by an iterative improvement in the policy. This
begins with an arbitrary policy $\Pi_0$, which is then improved via the bellman
equation. However, using a fixed policy prevents the introduction of the argmax
function which caused nonlinearity earlier. Thus the utility equation is as
follows $U_t(s) = R(s) + \Gamma\sum_{s'}T(s, \Pi_t(s), s')U_t(s')$. This
function $U_t = U^{\Pi_t}$. As the series of n equations is linear it can be
solved via linear algebra. The improvement step follows $\Pi_{t + 1} = \text{arg
max}_a \sum T(s,a,s')U_t(s')$. This process is known as policy iteration. The
time of the algorithm is typically $O(n^3)$. This process is guaranteed to
converge via the fact that there is a finite number of policies.

MDPs act as a foundation for reinforcement learning but are not reinforcmenet
learning in it of themselves.

\section{Reinforcement Learning}

There are two major versions of determining policies:
\begin{enumerate}
    \item Planning is the process of taking in a model and then returning a
    policy.
    \item Reinforcement learning is the process of being given a number of
    state transitions and then learning from the optimal policy given those
    transitions.
\end{enumerate}

Reinforcement learning is really reward maximization. RL is responsible for
strengthening the response to the stimulus.

We can further deconstruct the reinforcement learning into two smaller
components;
\begin{enumerate}
        \item Given a set of transitions a modeler builds a model from it.
        \item Giben a model a simulator generates a set of transitions from it.
\end{enumerate}

Depending on the algorithm chosen and the policies, simulating can actually
be highly expensive due to combinatorial explosion.

Value iteration and policy iteration are types of planning algorithms.

Model based reinforcement learning is defined as $$\text{transitions}
\rightarrow \text{modeler} \rightarrow \text{model} \rightarrow \rightarrow
\text{Planner} \rightarrow \text{policy}$$.

Reinforcement based planners are defined as $$\text{model} \rightarrow
\text{Simulator} \rightarrow \text{transitions} \rightarrow \text{Learner}
\rightarrow \text{policy}$$.

Reinforcement learning on policies directly is known as policy search. However,
it suffers from issues with the temporal credit assignment problem. That is,
because feedback is only recieved at the end of a series of actions and not
during that then it is extraordinarily difficult to determine the next best
action to take. This is overcome by changing the formulation of the problem
to MDPs.

RL  on the utility function is known as value-based approaches. It is
difficult to transfer utility into policies direct using the bellman
equations so itermediary methods are used.

RL using the transition functions are the rewards function is known as model
based RL. It is possible to use value iteration to solve the bellman equations.


Encoding in desired actions taken into the Bellman equations results in the
following function $Q(s,a) = R(s) + \gamma\sum_{s'}T(s,a,s')\text{max}_{a'}Q(s',
a')$. That is the equation maximizes the utility of the next step given a
specific state and action. It allows for the comparison of different actions.

Q learning is done by computing the $Q$ of given values, we can define $U(s) =
\text{max}_aQ(s,a)$ simply by always taking the best action and further we can
define $\Pi^*(s) = \text{argmax}_aQ(s,a)$. This technique where we solve for the
$Q$ is known as Q-learning which the evaluation of the Bellman equations from
data.

Q-learning is difficult as there is no immediate access to $T(s,a,s')$ or the 
reward function $R(s)$.  A transition is encoded as $<s,a,r,s'>$ where $s$ is 
the state, $a$ is the action taken, $r$ is the reward given for taking an action
and $s'$ is the state after the action is taken. We use an estimate of the 
$Q$ function which is defined $\hat{Q}(S,a) \xrightarrow{\alpha} r + 
\gamma\text{max}_{a'}\hat{Q}(s',a')$ where $\alpha$ is the learning rate. 
$\hat{Q}(s,a)$ can be considered the utility of the current state.

The learning rate shown as $v \xleftarrow{\alpha} x$ is actually represented as
$v \leftarrow (1 - \alpha)v + \alphax$.

Using a decaying learning rate, the supposition is that the $Q$ function
computes $E[r + \gamma\text{max}_{a'}Q(s',a')]$ which can then be written as
$R(s) + \gammaE_{s'}[\text{max}_{a'}Q(s',a')]$ which further reduces to $R(s) +
\gamma\sum_{s'}T(s,a,s')\hat{Q}(s',a')$. However due to the $\hat{Q}$ changing
over time, this only works as a generalized explanation and does not work as
a theorem of the convergence.

Given an arbitrary $\hat{Q}$, and the update function $\hat{Q}(s,a) \xleftarrow{
\alpha_t} r + \gamma\text{max}_{a'}\hat{Q}(s',a')$. Then using the update
function we have $\hat{Q}(s,a) \rightarrow Q(s,a)$ if $s,a$ are visited
infinitely often and $\sum_t \alpha_t = \inf$, $\sum_t \alpha_t^2 < \inf$, $s'
~ T(s,a,s')$ and $r ~ R(s)$. That is a the function uses a decaying learning
rate, next state $s'$ is drawn from the state transition function and the next
reward is drawn from the reward function. Then we have a solution to the
Bellman equations.

Q-learning is actually a family of different algorithms wherein the initial
$\hat{Q}$, decay rate of $\alpha$ and how actions $s'$ are chosen differently.
The policy of choosing actions is incredibly important for how the algorithm 
will behave. It is important to choose an action choice policy that is resistant
to the arbitrary $\hat{Q}$. The greedy policy (using $\hat{Q}$) finds the local
minima. 

There are several ways of overcoming the greedy minima. Simulated annealing
(that is taking a random action sometimes) can help to overcome these
difficulties. The resultant action choosing policy $\hat{\Pi}(s) = \text{argmax}
_a\hat{Q}(s,a)$ with probability $1 - \epsilon$ and $\hat{\Pi}(s) = \text{random
action}$ otherwise (that is probability $\epsilon$). MDP state-action pairs that 
don't link to any other aren't considered in the progress and the simulated 
annealing as they are unreachable. Therefore it is important to start in
the subgroup of states that leads to a final exit solution. However, this 
problem of finding a path is a complicated problem in it of itself and is a 
graph search problem with an exponential search factor.

If the exploration is greedy in the limit with infinite exploration it refers to
Q-learning with a decaying $\epsilon$ much like a decaying learning rate. Using 
GLE we find that $\hat{Q} \rightarrow Q$ and $\hat{\Pi} \rightarrow \Pi^*$. 

The explanation-exploitation dilemma wherein there is a trade-off between 
learning and use of that learning results from there being a singular agent
doing the exploitation and explanation. 

Encoding previous learning is possible for the learning algorithm and improve
its behavior. The EED is a fundamental tradeoff in RL. There is an inherent link
between model learning and planning. 

By setting the initial $\hat{Q}$ high, then the algorithm will search more
readily and is known as optimism in the face of uncertainty. 

Generalizing MDPs allows for function approximation.

\section{Game Theory}

Game theory can be defined as the mathematics of conflict. That is moving from 
single agents to multiple agents. 

Other agents can be considered in the transition model, however it is more 
effective to consider them explicitly.

A zero sum game is a game in which the sum of each reward state is always
constant. 

A strategy is a mapping of all possible states to actions.

Pure strategies are strategies that provide a complete definition of how a
player will play in a game.

The matrix form of a game encodes the games outcomes as a result of any and all
strategies played by all players and contains the complete outcome and
understanding of the entire game space. 

In a two player game, a player who goes first needs to consider the worst case
counter strategy. If the two players have opposite winning conditions where 
one's gain is another's loss. Minimax is the way in which the game is played out
by both players, the minimax strategy results in the game's value which is the
output cell of the minimax strategy. Formally in a 2-player, zero-sum
deterministic game of perfect information Minimax $\equiv$ Maximin and there 
always exists an optimal pure strategy for each player. 

In games, agents try to find the strategy that maximizes reward with the 
understanding that all other agents are also trying to maximize their reward at
that agents expense. 

Von Neumann's Theorem: For 2-player zero-sum non-deterministic game of perfect
information Minimax $\equiv$ Maximin and there exists an optimal pure strategy
for each player. 

A game of hidden information is one in which all of the states are not known
to each player. 

In games of hidden information Von Neumann's theorem breaks down and Maxamin $\n
eq$ Minimax.

Mixed strategies are used instead of pure strategies which can overcome the 
definition of consistency defined in perfect information games. A mixed strategy
is one where the optimal strategy is not a single option from the strategical 
combinations but of a probability distribution over the strategies.

Mixed strategies result in a distribution over the probability of $p$ wherein a
player picks strategies with probability $p$. The center of the game or the 
'value' of the game is the location wherein the strategical functions overlap.
There are cases where the functions of the probability distribution system do 
not overlap. In these cases the max of the lowest line is the center of the
game. 

2-player non-zero sum, non-deterministic games of hidden information have
several different types but one classical example is the Prisoner's Dilemma. 

The game generalizes into Nash Equilibrium, whereby $n$ players, with strategies
$S_1, S_2, \ldots S_n$ are the strategy sets from which players draw. The
situation wherein $S_1^* \in S_1, S_2^* \in S_2, \ldots S_n^* \in S_n$ are a
nash equilibrium if and only if $\forall_i S_I^* = \text{argmax}_{s_i} 
\text{utility}(S_1^*,ldots S_i^* \ldots S_n^*)$. The nash equilibrium refers to 
the idea that no player has any reason to change their strategy. Nash
equilibriums can be applied to pure and mixed strategies, the mixed version 
deals with probabilities over the strategy distributions. 

In the n-player pure strategy game, if elmination of strictly dominated 
strategies eliminates all but one combination, that combination is the 
unique Nash Equilibrium. Furthermore, in multi-round games elimination is 
done until no more is possible, then on the next round the process is continued
making NE an iterable process. 

Any NE strategies will survive iterated elinimation of strictly dominated 
strategies.

If $n$ is finite and $\forall_i S_i$ is finite then $\exists$ (mixed) NE. That
is there is always a NE of some variety. 

Iterated games create different dominant strategies than singular rounds 
wherein the dominant strategy may fall to a different NE than the initial 
setup of the NE. 

Symmetric games suffer from the tit-for-tat strategy more strongly than other
types of games.  

A way of evaluating the NE of a many step iterated game is to start 
at the last game and determine the value of that game, then to work your way
back over each game prior and then via proof by induction the games' values
are built back to the original game. Thus via an iterated game is the NE of a 
single game. We can do this via the 'sunk cost fallacy'. 

Given a game where the NE is $x$, $n$ repeated iterations of the game will 
result in $n$ repeated same NEs. This theorem holds for a single NE. 

The matrices of the games includes all representative information encoded with
it of the future, past and present. Re-encoding of the game can be done to 
cause reevaluation of the game conditions which causes the values of certain
game cells. 

Mechanism design is the process of evaluating and designing incentives to
get particular behavior, ie re-encoding the game. 

\section{More Game Theory}

A generalization of MDPs and iterative games is stochastic games (Markov) 
games and provide a model for multiagent RL. 

The NE of a SG (stochastic game) is a pair of policies where neither wants to
deviate. 

Stochastic games are invented by Shapley and encode the following items:
\begin{itemize}
	\item States $S$
	\item Actions for player $i$, $A_i$
	\item Transitions $T(s, (a,b), s')$
	\item Rewards for player $i$, $R_i$ where $R_1(s,(a,b)), R_2(s,(a,b))$
	\item Discount $\gamma$
\end{itemize}

Differing definitions of MDPs and SG include the discount  others however do not
include them. 

SG is a generalization of MDPs. 

Zero-sum SG is based can be extrapolated from MDPs and we can use a generalized
version of the Bellman equation to accomplish this. 

We begin with the Bellman equation $\Q_i^*(s,(a,b)) = R_i(s_i(a,b)) + \gamma
\sum_{s'} T(s,(a,b),s')\text{max}_{a',b'}Q_i^*(s',(a',b'))$. The problem with
the initial formulation for SG is that the $Q^*_i(s',(a',b'))$  assumes that
every joint action is designed to benefit the maximizer, which does not hold for
SG (holds for single player games). 

		
\end{document}
